\begin{itemize}
	\item In this section, we will present a number of experiments comparing the performance of \Nystrom and RFF.  We show that under a fixed memory budget, RFF outperforms \NystromNS, and this is in spite of the fact that \Nystrom often attains lower kernel approximation error (Frobenius/spectral).
	\item we will explain this using fixed design theory, and the $\Delta$ parameter from \citep{avron17}.
\end{itemize}

\subsection{\Nystrom vs. RFF: Empirical comparison}
\begin{itemize}
	\item Across various classification/regression datasets (TIMIT, yearpred, CovType, Census, Adult), we compare \Nystrom vs. RFF, for a wide range of numbers of features.
	\item We compare these two methods in various ways: 1) For a fixed \# of features, 2) For a fixed amount of memory, 3) for a fixed kernel approximation error (x-axis = spectral/Frobenius norm).  For the y-axis in these plots, we consider kernel approximation error, and downstream performance.
	\item We observe that RFF outperforms \Nystrom for a fixed memory budget, and for a fixed kernel approximation error.
\end{itemize}

\subsection{\Nystrom vs. RFF: ``$\Delta$-analysis''}
\begin{itemize}
	\item Across a few small datasets (Census, Adult), we measure the $\Delta$ parameter, and plot the performance of \Nystrom vs. RFF in terms of this parameter.
	\item We observe that \Nystrom and RFF models that have similar $\Delta$ values, have similar generalization performance.  We show that this correlation is much stronger than the correlation between kernel approximation error and generalization performance.
	\item We plot the value of $\Delta$ attained by \Nystrom and RFF as a function of memory and number of features.  We show that for a fixed memory budget, RFF attains much smaller values of $\Delta$.
\end{itemize}