\begin{itemize}
	\item Give an overview of kernel approximation generalization bounds.  Note that the central object they generally use to bound the generalization error is $\|K-\tK\|$.  Discuss the more recent results which approach the problem differently \avner{We should read recent papers closely.}
	\item Discuss Nystrom vs. RFF paper.  Our main differentiators: 1) we consider larger/more challenging datasets, 2) we consider much larger numbers of features, 3) we consider memory budget.
	\item Discuss low-precision, and recent interest in doing low-precision training.
	\item Discuss fastfood/circulant RFF, and other attempts at reducing memory footprint of projection matrix.  Note that this doesn't address the memory of the features themselves, which can take a ton of space for moderate size mini-batches. Furthermore, our method can be combined with these methods in order to reduce the memory footprint even further (in fact, that is what we do!).
	\item Discuss other attempts at improving RFF (orthogonal RFF, Spherical Structured Feature Maps, Gaussian quadrature, quasi-monte carlo, etc.), and why we are different.  Discuss that while these methods often improve the kernel approximation error for a fixed number of features, they do not increase the rank for a fixed amount of memory.
	\item Perhaps discuss methods of improving Nystrom?  Sampling distributions, etc.
\end{itemize}