\begin{theorem}
  Suppose that $\norm{K} \geq \lambda$ and $\delta^2_b \leq \lambda$.
  Then for any $1/2 \geq \Delta_0 \geq \sqrt{\frac{2n/\lambda}{m}} + \frac{4n/\lambda}{3m}$, with $\Delta =
  \Delta_0 + \delta^2_b / \lambda$,
  \begin{equation*}
    P((1 - \Delta)(K + \lambda I_n) \preceq (Z + C) (Z + C)^\top + \lambda I_n \preceq (1 + \Delta)(K + \lambda I_n)) \geq 1 - 16 \tr((K +
    \lambda I_n)^{-1} (K + \delta^2_b I_n)) \exp \left( -\frac{3m \Delta_0^2}{16n/\lambda} \right).
  \end{equation*}
  Thus if we use $m \geq \frac{16}{3 \Delta_0^2} n/\lambda \log (16 \tr((K + \lambda I_n)^{-1} (K +
  \delta^2_b I_n)) / \rho)$
  features, then $(Z + C) (Z + C)^\top + \lambda I_n$ is a $\Delta$-spectral approximation of $K + \lambda I_n$,
  with probability at least $1 - \rho$.
\end{theorem}

\begin{proof}
  Consider the condition $(1 - \Delta)(K + \lambda I_n) \preceq (Z + C) (Z + C)^\top + \lambda I_n \preceq (1 +
  \Delta)(K + \lambda I_n)$.
  Let $B \defeq (K + \lambda I_n)^{-1/2}$, which is symmetric, then $\norm{B}^2 =
  \norm{K + \lambda I_n}^{-1} \leq 1/\lambda$.
  Conjugating by $B \defeq (K + \lambda I_n)^{-1/2}$ (i.e.\ multiplying by $B$ on the
  left and right) yields an equivalent condition (as $B$ is
  invertible):
  \begin{equation*}
    (1 - \Delta) I_n \preceq (K + \lambda I_n)^{-1/2} ((Z + C) (Z + C)^\top + \lambda I_n) (K + \lambda I_n)^{-1/2} \preceq (1 +
    \Delta) I_n.
  \end{equation*}
  Subtracting $I_n$ from both sides:
  \begin{equation*}
    -\Delta I_n \preceq (K + \lambda I_n)^{-1/2} ((Z + C) (Z + C)^\top + \lambda I_n - (K + \lambda I_n)) (K + \lambda
    I_n)^{-1/2} \preceq \Delta I_n.
  \end{equation*}
  This is equivalent to
  \begin{equation*}
    \norm{(K + \lambda I_n)^{-1/2} ((Z + C) (Z + C)^\top - K) (K + \lambda I_n)^{-1/2}} \leq \Delta.
  \end{equation*}
  Suppose that $\norm{B((Z + C)(Z + C)^\top - (K + D))B} \leq \Delta_0$.
  Then since $0 \preceq D \preceq \delta^2_b I_n$, $\norm{B(K + D - K) B} = \norm{B D B} \leq \delta^2_b
  \norm{B}^2$, and the triangle inequality yields
  \begin{align*}
    \norm{B((Z + C)(Z + C)^\top - K)B}
    &\leq \norm{B((Z + C)(Z + C)^\top - (K + D))B} + \norm{B (K + D - K)B} \\
    &\leq \Delta_0 + \delta^2_b \norm{B}^2 \\
    &\leq \Delta_0 + \delta^2_b/\lambda \\
    &= \Delta.
  \end{align*}
  Thus it suffices to show that
  \begin{equation*}
    P(\norm{B((Z + C)(Z + C)^\top - (K + D))B} \geq \Delta_0) \leq 8 \tr((K + \lambda I_n)^{-1} (K +
    \delta^2_b I)) \exp \left( -\frac{3m \Delta_0^2}{16n/\lambda} \right).
  \end{equation*}

  We have $L \defeq 2n \norm{B}^2 \leq 2n/\lambda$.
  Let $\lambda_1, \dots, \lambda_n$ be the eigenvalues of $K$.
  We have
  \begin{equation*}
    M \defeq B(K + \delta^2_b I_n) B =  (K + \lambda I_n)^{-1/2} (K + \delta^2_b I_n) (K + \lambda
    I_n)^{-1/2} = \diag((\lambda_1 + \delta^2_b)/(\lambda_1 + \lambda), \dots,
    (\lambda_n + \delta^2_b) / (\lambda_n + \lambda)).
  \end{equation*}
  By assumption, $\norm{K} \geq \lambda$, so $\norm{M} = \frac{\lambda_1 + \delta^2_b}{\lambda_1 + \lambda} \geq
  1/2$.
  We also assume that $\delta^2_b \leq \lambda$, so $\norm{M} \leq 1$.
  Moreover, $\tr(M) = \tr((K + \lambda I_n)^{-1/2} (K + \delta^2_b I) (K + \lambda I_n)^{-1/2}) =
  \tr((K + \lambda I_n)^{-1} (K + \delta^2_b I_n))$.
  The condition of $\Delta_0 \geq \sqrt{L \norm{M} / m} + 2L/3m$ becomes $\Delta_0 \geq \sqrt{\frac{2n/\lambda}{m}} + \frac{2n/\lambda}{3m}$.
  The bound of Proposition~\ref{prop:quantized_concentration} becomes:
  \begin{align*}
    &P(\norm{(K + \lambda I_n)^{-1/2} ((Z + C) (Z + C)^\top - (K + D)) (K + \lambda
      I_n)^{-1/2}} \geq \Delta_0) \\
    \leq&\ \frac{8 \tr((K + \lambda I_n)^{-1} (K + \delta^2_b I_n))}{1/2} \exp \left( -\frac{m
      \Delta_0^2}{4\frac{n}{\lambda} (1 + 2\Delta_0/3)} \right) \\
    \leq&\ 16 \tr((K + \lambda I_n)^{-1} (K + \delta^2_b I_n)) \exp \left( -\frac{3m\Delta_0^2}{16n/\lambda} \right)
  \end{align*}
  where we get the last inequality from the assumption that $\Delta_0 \leq 1/2$.

  Letting this probability be $\rho$ and solving for $m$ yields
  \begin{equation*}
    m \geq \frac{16}{3 \Delta_0^2} n/\lambda \log (16 \tr((K + \lambda I_n)^{-1} (K + \delta^2_b I_n)) / \rho).
  \end{equation*}

\end{proof}


There is a bias--variance trade-off: as we decrease the number of bits $b$, under
a fixed memory budget, we can use more features, and $(Z + C)(Z + C)^\top$
concentrates more strongly (lower variance) around the expectation $K + D$ with
$0 \preceq D \preceq \delta^2_b I_n$, but this expectation is further away from the true kernel
matrix $K$ (larger bias).
Thus there should be an optimal number of bits $b^*$ that balances the bias and
the variance.

As a sanity check, if we let the number of bits $b$ goes to $\infty$, we recover the
result of \citet{avron17}.
\begin{corollary}
	Suppose that $\norm{K} \geq \lambda$.
	Then for any $1/2 \geq \Delta \geq \sqrt{\frac{2n/\lambda}{m}} + \frac{4n/\lambda}{3m}$,
	\begin{equation*}
	P((1 - \Delta)(K + \lambda I_n) \preceq Z Z^\top + \lambda I_n \preceq (1 + \Delta)(K + \lambda I_n)) \geq 1 - 16 \tr((K +
	\lambda I_n)^{-1} K) \exp \left( -\frac{3m \Delta^2}{16n/\lambda} \right).
	\end{equation*}
	Thus if we use $m \geq \frac{16}{3 \Delta^2} n/\lambda \log (16 \tr((K + \lambda I_n)^{-1} K) / \rho)$
	features, then $Z Z^\top + \lambda I_n$ is a $\Delta$-spectral approximation of $K + \lambda I_n$,
	with probability at least $1 - \rho$.
\end{corollary}
The constants are slightly different from that of \citet{avron17} as we use the
real features $\sqrt{2} \cos(w^T x + b)$ instead of the complex features $\exp(i
w^T x)$, and our definition of $\Delta$-spectral approximation is different from theirs.

The number of features depend linearly on $n/ \lambda$.
\citet{avron17} provided a lower bound, showing that the number of random Fourier features
must depend linearly on $n / \lambda$.
For optimal minimax rate, the value of $\lambda$ is of order $\sqrt{n}$ (\todo{check
	this}), so the number of features is still sublinear in $n$.